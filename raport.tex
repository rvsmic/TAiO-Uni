\documentclass[a4paper, 12pt]{article}

\usepackage[utf8]{inputenc}
\usepackage{graphicx}
\usepackage{tikz}
\usepackage{booktabs}
\usepackage[MeX]{polski}
\usepackage[linesnumbered]{algorithm2e}
\usepackage{listings}
\usepackage{amsmath}
\usepackage{amssymb}
\usepackage[OT4]{fontenc}
\usepackage{verbatim}
\usepackage[all]{nowidow}
\usepackage{indentfirst}
\usepackage{hyperref}
\usepackage{url}


\title{Zadanie laboratoryjne: raport}
\author{Michał Rusinek}
\date{\today}

\begin{document}

\maketitle
\thispagestyle{empty}
\newpage

\
\thispagestyle{empty}
\newpage


\tableofcontents
\thispagestyle{empty}
\newpage

\section*{Wprowadzenie}
\addcontentsline{toc}{section}{Wprowadzenie}

Celem zadania było zrealizowanie i omówienie analizy grafów poprzez implementację programów komputerowych oraz przedstawienie sprawozdania zawierającego opis użytych metod.

Wybranym wariantem zadania było stworzenie programu rozwiązującego polecenie dla grafów skierowanych.

Zespół składa się z jednej osoby.

\section{Opis techniczny programu}

Program został napisany w języku Python z wykorzystaniem biblioteki NetworkX do analizy grafów. Pozwala ona na łatwe tworzenie, manipulację i analizę grafów, a także zawiera wiele algorytmów do analizy sieci.

Instrukcja kompilacji i uruchomienia programu znajduje się w pliku\\
\texttt{instrukcja\_uruchomienia.pdf}.

\section{Opis danych wejściowych}

Dane wejściowe składają się z pliku tekstowego zawierającego opisy jednego lub więcej grafów. Każdy opis grafu rozpoczyna się od liczby wierzchołków, a następnie zawiera wiersze macierzy sąsiedztwa, które reprezentują krawędzie między wierzchołkami. Każdy graf w pliku jest oddzielony pustą linią. Pierwsza linia pliku zawiera liczbę grafów, ale w przypadku kiedy plik zawiera jeden graf w pierwszej linii znajduje się od razu liczba wierzchołków.

\section{Opis algorytmów}

\subsection{Rozmiar grafu}

\subsubsection*{Definicja}

Rozmiar grafu jest definiowany jako liczba wszystkich krawędzi w grafie. Dla macierzy sąsiedztwa grafu skierowanego, rozmiar grafu obliczamy jako sumę wszystkich elementów macierzy. \cite{rozmiar_grafu}

\subsubsection*{Algorytm}

Jako iż zgodnie z założeniami rozpatrywane są wyłącznie grafy skierowane - przejście przez każdy wiersz macierzy i sumowanie wartości.

\subsubsection*{Kod rozwiązania}

\begin{verbatim}
def calculate_size(graph):
    return sum([sum(row) for row in graph])
\end{verbatim}

\subsubsection*{Złożoność obliczeniowa}

$O(V^2)$, gdzie V to liczba wierzchołków w grafie.

\subsection{Metryki w zbiorze wszystkich grafów}

\subsubsection{Gęstość grafu}

\subsubsection*{Definicja}

Gęstość grafu to stosunek liczby krawędzi w grafie do liczby wszystkich możliwych krawędzi. Dla grafu skierowanego, liczba wszystkich możliwych krawędzi wynosi \(V^2\), gdzie V to liczba wierzchołków. \cite{gestosc_grafu}

\subsubsection*{Algorytm}

Algorytm oblicza gęstość grafu jako stosunek rozmiaru grafu do liczby wszystkich
możliwych krawędzi.

\subsubsection*{Kod rozwiązania}

\begin{verbatim}
def density(graph):
    g = nx.DiGraph()
    for row_idx, row in enumerate(graph):
        for col_idx, cell in enumerate(row):
            if cell == 1:
                g.add_edge(row_idx, col_idx)
    return nx.density(g)
\end{verbatim}

\subsubsection*{Złożoność obliczeniowa}

$O(V^2)$, gdzie V to liczba wierzchołków w grafie.

\subsubsection{S-metryka grafu}

\subsubsection*{Definicja}

S-metryka mierzy, jak mocno połączone są kluczowe węzły (huby) w sieci. Pozwala ona lepiej różnicować sieci o podobnym rozkładzie stopni poprzez ocenę, jak dobrze połączone są główne węzły o wysokim stopniu. \cite{s_metryka}

\subsubsection*{Algorytm}

Algorytm wykorzystuje podejście polegające na zliczaniu trójkątów (czyli cykli o długości 3) w grafie. S-metryka mierzy liczbę trójkątów w stosunku do możliwej liczby trójkątów w grafie losowym o tej samej liczbie wierzchołków i krawędzi.

\subsubsection*{Kod rozwiązania}

\begin{verbatim}
def s_metric(graph):
    g = nx.DiGraph()
    for row_idx, row in enumerate(graph):
        for col_idx, cell in enumerate(row):
            if cell == 1:
                g.add_edge(row_idx, col_idx)
    return nx.s_metric(g)
\end{verbatim}

\subsubsection*{Opis szczegółowy}

\begin{enumerate}
    \item Tworzenie grafu skierowanego \texttt{g} przy użyciu biblioteki NetworkX.
    \item Dodawanie krawędzi do grafu \texttt{g} na podstawie macierzy sąsiedztwa.
    \item Obliczanie s-metryki grafu za pomocą funkcji \texttt{nx.s\_metric}, która:
    \begin{itemize}
        \item Iteruje przez wszystkie możliwe trójkąty w grafie.
        \item Dla każdego trójkąta sprawdza, czy jest zamknięty (czyli czy \\wszystkie trzy krawędzie są obecne).
        \item Oblicza stosunek liczby trójkątów do liczby wszystkich możliwych trójkątów.
    \end{itemize}
\end{enumerate}

\subsubsection*{Złożoność obliczeniowa}

Obliczenie s-metryki ma złożoność $O(V^3)$ dla grafów gęstych, gdzie V to liczba wierzchołków.

\subsection{Maksymalny cykl w grafie, liczba maksymalnych cykli}

\subsubsection*{Definicja}

Cykl w grafie to ścieżka, która zaczyna się i kończy w tym samym wierzchołku. Maksymalny cykl w grafie to cykl, który ma największą liczbę wierzchołków. Liczba maksymalnych cykli to liczba cykli o tej samej długości. \cite{cykle}

\subsubsection*{Algorytm}

Algorytm wykorzystuje funkcję \texttt{nx.simple\_cycles} z biblioteki NetworkX do znajdowania wszystkich prostych cykli w grafie. Następnie oblicza długość maksymalnego cyklu oraz liczbę maksymalnych cykli.

Proste cykle, znane również jako elementarne cykle, to zamknięte ścieżki, w których każdy węzeł odwiedzany jest tylko raz. W grafach skierowanych dwa cykle są różne, jeśli nie są cyklicznymi permutacjami siebie nawzajem. W grafach nieskierowanych dwa cykle są różne, jeśli nie są ani cyklicznymi permutacjami siebie nawzajem, ani odwróceniem jednego z nich.

W przypadku, gdy długość cykli nie jest ograniczona, stosujemy nierekurencyjną wersję algorytmu Johnsona, wykorzystującą iteratory i generatory. Natomiast dla cykli o ograniczonej długości, używamy algorytmu Gupty i Suzumury. Istnieją również inne algorytmy, które mogą być bardziej efektywne w niektórych przypadkach.

Aby poprawić efektywność tych algorytmów, stosujemy techniki przetwarzania wstępnego. W grafach skierowanych skupiamy się na silnie spójnych składowych grafu, generując wszystkie proste cykle zawierające dany węzeł. Następnie usuwamy ten węzeł i dzielimy resztę grafu na silnie spójne składowe. W grafach nieskierowanych koncentrujemy się na dwuspójnych składowych, generując proste cykle zawierające daną krawędź. Po usunięciu tej krawędzi, również dzielimy resztę grafu na dwuspójne składowe.

W ten sposób możemy skutecznie znaleźć i analizować proste cykle w grafie, co pozwala lepiej zrozumieć strukturę analizowanej sieci.\cite{s_cycles}

\subsubsection*{Kod rozwiązania}
\begin{verbatim}
def find_max_cycles(graph):
    g = nx.DiGraph()
    for row_idx, row in enumerate(graph):
        for col_idx, cell in enumerate(row):
            if cell == 1:
                g.add_edge(row_idx, col_idx)
    cycles = list(nx.simple_cycles(g))
    max_cycle_len = max(
        [len(cycle) for cycle in cycles]
    ) if cycles else 0
    max_cycles = [
        cycle for cycle in cycles if len(
            cycle
        ) == max_cycle_len
    ]
    return max_cycle_len, max_cycles
\end{verbatim}

\subsubsection*{Opis szczegółowy}

\begin{enumerate}
    \item Tworzenie grafu skierowanego \texttt{g} przy użyciu biblioteki NetworkX.
    \item Dodawanie krawędzi do grafu \texttt{g} na podstawie macierzy sąsiedztwa.
    \item Znajdowanie wszystkich prostych cykli w grafie \texttt{g} za pomocą funkcji \texttt{nx.simple\_cycles}, która:
    \begin{itemize}
        \item Iteruje przez wszystkie możliwe cykle w grafie.
        \item Sprawdza każdy cykl, czy jest prosty (czyli nie zawiera podcykli).
        \item Zwraca listę wszystkich prostych cykli.
    \end{itemize}
    \item Obliczanie długości maksymalnego cyklu oraz liczby maksymalnych cykli.
\end{enumerate}

\subsubsection*{Złożoność obliczeniowa}

Złożoność algorytmu do znajdowania prostych cykli wynosi \\$O((V+E)(C+1))$, gdzie $V$ to liczba wierzchołków, $E$ to liczba krawędzi, a $C$ to liczba prostych cykli. W najgorszym przypadku złożoność wynosi $O(V^3)$.

\section{Wyniki testów obliczeniowych}

Testy zostały przeprowadzone na kilku przykładowych grafach. Poniżej przedstawiono przykładowe wyniki dla jednego z grafów:

\subsection{Przykładowy test 1}

\subsubsection*{Dane wejściowe}

\begin{verbatim}
4
0 1 0 0
0 0 1 0
0 0 0 1
0 1 0 0
\end{verbatim}

\subsubsection*{Wyniki}

\begin{verbatim}
Graph 1
  -> Nodes: 4
  -> Size: 4
  -> Density:  0.33
  -> S-metric: 19.00
  -> Max cycle length: 3
  -> Max cycle count: 1
\end{verbatim}

\subsubsection*{Czas wykonania}

Czas wykonania programu dla powyższego grafu wyniósł 0.177 sekundy.

\subsection{Przykładowy test 2}

\subsubsection*{Dane wejściowe}

\begin{verbatim}
2
13
0 0 1 1 1 1 0 0 1 0 0 1 0
0 0 0 1 1 1 1 1 1 1 0 0 0
0 0 0 0 1 0 0 1 1 0 1 0 1
0 0 1 0 0 1 0 1 0 0 1 0 0
1 1 1 0 0 0 0 1 1 0 0 0 0
1 0 0 1 1 0 1 0 1 0 0 0 1
0 0 1 0 0 1 0 0 0 0 1 1 1
1 0 0 0 1 1 0 0 1 1 1 0 1
1 1 1 0 1 0 1 0 0 0 0 0 0
1 0 1 0 1 0 1 1 0 0 1 1 0
1 0 1 0 1 0 0 1 0 0 0 0 1
0 1 1 1 1 0 1 0 0 1 0 0 0
0 0 0 1 1 0 1 0 0 0 0 0 0

13
0 0 1 0 0 1 1 1 1 0 0 0 0
1 0 1 1 1 1 0 0 1 0 1 1 0
1 1 0 1 0 1 0 1 1 0 1 1 0
1 1 1 0 1 1 1 1 0 0 1 0 0
1 0 1 1 0 0 0 0 1 0 0 0 1
1 1 0 1 0 0 1 0 1 0 1 0 1
0 1 0 1 0 1 0 0 0 0 1 1 1
1 0 1 0 1 0 1 0 0 0 1 0 1
1 1 0 0 0 0 1 1 0 1 0 1 0
1 1 0 0 1 0 1 1 1 0 1 0 0
0 1 0 1 1 0 1 1 0 1 0 0 0
0 1 0 0 0 0 0 1 0 0 1 0 1
0 0 0 1 1 1 0 0 1 0 0 1 0
\end{verbatim}

\subsubsection*{Wyniki}

\begin{verbatim}
Graph 1
 -> Nodes: 13
 -> Size: 71
 -> Density:  0.46
 -> S-metric: 8956.00
 -> Max cycle length: 13
 -> Max cycle count: 10058

 ----------------- 

Graph 2
 -> Nodes: 13
 -> Size: 81
 -> Density:  0.52
 -> S-metric: 13269.00
 -> Max cycle length: 13
 -> Max cycle count: 67039
 \end{verbatim}

\subsubsection*{Czas wykonania}

Czas wykonania programu dla powyższych grafów wyniósł 1.83 sekundy.

\subsection{Przykładowy test 3}

\subsubsection*{Dane wejściowe}

\begin{verbatim}
2
4
0 1 0 0
0 0 1 0
0 0 0 1
0 1 0 0

3
0 1 0
0 0 1
0 0 0
\end{verbatim}

\subsubsection*{Wyniki}

\begin{verbatim}
Graph 1
  -> Nodes: 4
  -> Size: 4
  -> Density:  0.33
  -> S-metric: 19.00
  -> Max cycle length: 3
  -> Max cycle count: 1
   
----------------- 
   
Graph 2
  -> Nodes: 3
  -> Size: 2
  -> Density:  0.33
  -> S-metric:  4.00
  -> Max cycle length: 0
  -> Max cycle count: 0
\end{verbatim}

\subsubsection*{Czas wykonania}

Czas wykonania programu dla powyższego grafu wyniósł 0.243 sekundy.

\section*{Wnioski}
\addcontentsline{toc}{section}{Wnioski}

Na podstawie przeprowadzonych testów można wyciągnąć następujące wnioski:

\begin{itemize}
    \item Program poprawnie analizuje grafy skierowane, obliczając ich rozmiar, gęstość, s-metrykę oraz maksymalne cykle.
    \item Wyniki testów pokazują, że program jest w stanie przetwarzać zarówno małe, jak i duże grafy, zachowując przy tym akceptowalny czas wykonania.
    \item Algorytmy zaimplementowane w programie mają złożoność obliczeniową odpowiednią do analizy grafów o różnej wielkości, co czyni program skalowalnym.
    \item Biblioteka NetworkX okazała się być efektywnym narzędziem do analizy grafów, umożliwiając szybkie i łatwe implementowanie zaawansowanych algorytmów grafowych.
    \item Program może być rozszerzony o dodatkowe funkcjonalności, takie jak analiza multigrafów czy implementacja algorytmów do znajdowania cykli Hamiltona, co zwiększyłoby jego użyteczność w różnych dziedzinach.
\end{itemize}

Podsumowując, zrealizowane zadanie umożliwia efektywną analizę grafów skierowanych i może być użyte w różnych zastosowaniach, takich jak analiza sieci społecznych, struktury biologiczne czy optymalizacja sieci.

\newpage
\addcontentsline{toc}{section}{Literatura}
\bibliographystyle{plain}
\bibliography{bibliografia}

\end{document}