\documentclass[a4paper, 12pt]{article}

\usepackage[utf8]{inputenc}
\usepackage{graphicx}
\usepackage{tikz}
\usepackage{booktabs}
\usepackage[MeX]{polski}
\usepackage[linesnumbered]{algorithm2e}
\usepackage{listings}
\usepackage{amsmath}
\usepackage{amssymb}
\usepackage[OT4]{fontenc}
\usepackage{verbatim}
\usepackage[all]{nowidow}
\usepackage{indentfirst}
\usepackage{hyperref}
\usepackage{url}


\title{Zadanie laboratoryjne: instrukcja uruchomienia}
\author{Michał Rusinek}
\date{\today}

\begin{document}

\maketitle
\thispagestyle{empty}
\newpage

\section{Wymagania}
Do uruchomienia programu potrzebne są (użyte podczas tworzenia):
\begin{itemize}
    \item Python 3.13
    \item biblioteka NetworkX 3.4.2
\end{itemize}

\section{Uruchomienie}
Program uruchamiamy za pomocą polecenia:
\begin{verbatim}
    python projekt.py <plik_wejściowy>
\end{verbatim}

\end{document}